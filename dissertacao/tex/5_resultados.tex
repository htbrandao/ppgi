% \chapter{Resultados}
\label{Cap:Resultados}

Neste capítulo serão apresentados os detalhes da implementação, resultados, métricas de desempenho e a metodologia utilizada na avaliação deste trabalho.\\

\section{Cenários Modelados}\label{sec:implementacao}

Esta seção detalha nossa implementação para o efetuar o reconhecimento da intensidade emoções na fala em nosso idioma nativo. Utilizando aprendizagem supervisionada e não supervisionada para uma tarefa de reconhecimento da intensidade da emoção expressa vocalmente em português.

Para tanto, em virtude do desafio da demanda, ratificado pela escassez de dados, nos aproveitamos da Fusão de Dados para criar, primeiramente, uma solução não supervisionada que consiga extrair características que sejam representativas o suficiente para que, mesmo com a dimensionalidade reduzida em comparação a original, consigamos reconstruir o dado de entrada; e posteriomente utilizar esses dados comprimidos, constituído de atributos suficientemente relevantes, para desenvolver um modelo supervisionado de inferência da intensidade.

Definida a arquitetura composta por um \acrlong{AE} e um classificador, tomamos apenas as emoções comuns entre as duas bases de dados (alegria, medo, raiva e surpresa), totalizando 1364 registros, onde 51\% apresentam a o \textit{label} relativo a intensidade. %A Figura \ref{fig:design} apresenta o \textit{design} do \textit{BRAVO}.

% \begin{figure}[h]
%     \centering
%     \includegraphics[width=1.0\textwidth]{img/design.PNG}
%     \caption{\label{fig:design}\textcolor{blue}{\textit{Design} da proposta para o \textit{BRAVO}}}
% \end{figure}

Foram realizados dois cenários experimentais, utilizando 64 e 128 \acrshort{MFCC}s, respectivamente. Em ambos, o \acrlong{AE} foi treinado com dados de $X$ e, em seguida, treinamos uma rede neural densa multicamada para classificar a intensidade, treinada e validada apenas na porção dos dados resultantes da aplicação de $f$ em $X_{VIVAE}$, uma vez que $X_{VERBO}$ não tem correspondência com o contradomínio ($Z$) das intensidades.

Uma vez que não há \textit{label} correspondente às intensidades par $X_{VERBO}$, devemos investigar se há algum sentido nos resultados quando aplicarmos a classificação a esses dados, que até então ainda não foram vistos pelo classificador.

Assim, faz-se necessária uma forma de analisar os registos de $X_{VIVAE}$ e$X_{VERBO}$ quanto às intensidades e a predição dessas intensidades, respectivamente. Podemos utilizar o \acrshort{PCA} para reduzir a dimensionalidade dos registros e observar o comportamento da classificação.

A utilização de \acrshort{PCA} encontra registros na literatura, tanto de \acrshort{ML} e \acrshort{DL} aplicados a voz e emoções. Temos~\cite{pca1} que utilizou o \acrshort{PCA} para gerar \textit{features} de um modelo, observando que modelos treinados com as \textit{features} obtidas a partir do \acrshort{PCA} mantiveram sua performance quando comparados aos treinados com outros atributos;~\cite{pca2} que apesar de fazer uma \textit{feature selection} prévia, optou por utilizar o \acrshort{PCA} para reduzir em quase três vezes a dimensionalidade do dado para conseguir uma visualização bidimensional dos registros;~\cite{pca3} que buscou desenvolver um sistema de reconhecimento de emoções para áudios com ruídos, utilizando dados $64$-dimensionais que seriam reduzidos para um espaço $6$-dimensional utilizando e comparando técnicas distintas, sendo uma delas o \acrshort{PCA}; e~\cite{pca5} que aproxima o grau de \textit{encoding} de um \acrlong{textit{AEutoencoder} linear - em seu espaço latente $n$-dimensional - ao de um \acrshort{PCA} com $n$ componentes.

Então, sabemos  da utilização do \acrshort{PCA} tanto para permitir a visualização de dados através da redução de dimensionalidade quanto para gerar as \textit{features} utilizadas em modelos de reconhecimento de emoções, o que nos diz que o \acrshort{PCA} seria uma solução adequada para reduzir a dimensionalidade dos dados enquanto preserva características relevantes das amostras.


% \section{Experimentos}

% Nesta seção apresentaremos, detalhadamente, os resultados dos dois experimentos realizados. Ambos foram submetidos à mesma metodologia e aferição de métrica, o que nos possibilita comparar diretamente o desempenho oriundo das execuções.\\

\section{Resultados dos experimentos}

% \subsection{Primeiro experimento: 64 \textit{MFCC}s}

A Figura \ref{fig:ae64} apresenta a arquitetura do modelo Autoencoder, onde podemos observar que os vetores do espaço latente, portanto, o vetor comprimido para ser posteriormente reconstruído, terão $32$ posições.

Para este modelo, o valor final da \textit{Loss} média ao aplicá-lo ao seu conjunto de teste foi de $6,40$. A Figura \ref{fig:predae64} ilustra uma predição feita pelo Autoencoder: A linha azul representa o dado original (\textit{input});a linha tracejada laranja representa a predição (\textit{output}) daquele registro; e a linha verde foi calculada tomando a diferença entre os coeficientes do \textit{input} e do \textit{output}.

% --- PRED AE 64 --- %
\begin{figure}[h]
    \centering
    \includegraphics[width=0.6\textwidth]{img/_predae64.png}
    \caption{\label{fig:predae64}Exemplo de predição (\textit{encoding e decoding}) do \textit{Autoencoder} para 64 \textit{MFCC}s}
\end{figure}

A Figura \ref{fig:clf64} apresenta a arquitetura do modelo de classificação de intensidade, enquanto a Tabela \ref{table:metricasclf64} apresenta métricas de desempenho para sua aplicação no conjunto de teste. Podemos observar que o melhor resultado para o \textit{F1-Score} deu-se para a intensidade Fraca, com o valor de $0,68$, e ficando abaixo de $0,5$ apenas para a classe Forte.

Aplicamos o classificador aos dados oriundos do $X_{VERBO}$ e anotamos o resultado. Então efetuamos um \textit{PCA} com 2 componentes ao resultado do \textit{encoding} de $X$ (Figura \ref{fig:pca64}), onde as cores representam a classe - original para os dados do $VIVAE$ e predição do classificador para os dados do $VERBO$ - e podemos observar a formação de 2 grandes \textit{clusters}, um a esquerda e outro a direita.

Percebemos através dos \textit{labels}, que estes \textit{clusters} conseguiram distinguir bem as duas bases de dados, bem como um movimento descendente de aumento da intensidade da emoção em ambos os \textit{clusters}. Temos uma visualização mais simples na Figura \ref{fig:pca64-2}, onde agrupamos as classes Fraca e Moderada em uma nova classe denominada Baixa e as classes Forte e Pico em uma classe denominada Alta.

% --- CLF 64 Metrics --- %
\begin{table}[h]\caption{\label{table:metricasclf64}Métricas de avaliação para o classificador do experimento para 64 \textit{MFCCs}}
    \centering
    \begin{tabular}{l|ccc|}
        \cline{2-4}
                                      & \multicolumn{3}{c|}{Métricas}                                                               \\ \hline
            \multicolumn{1}{|l|}{Intensidade} & \multicolumn{1}{c|}{Precision}      & \multicolumn{1}{c|}{Recall}           & F1-Score      \\ \hline
            \multicolumn{1}{|l|}{Fraca}       & \multicolumn{1}{c|}{\textbf{0,73}}  & \multicolumn{1}{c|}{0,64}             & \textbf{0,68} \\ \hline
            \multicolumn{1}{|l|}{Moderada}    & \multicolumn{1}{c|}{0,53}           & \multicolumn{1}{c|}{0,52}             & 0,53          \\ \hline
            \multicolumn{1}{|l|}{Forte}       & \multicolumn{1}{c|}{0,46}           & \multicolumn{1}{c|}{0,50}              & 0,48          \\ \hline
            \multicolumn{1}{|l|}{Pico}        & \multicolumn{1}{c|}{0,64}           & \multicolumn{1}{c|}{\textbf{0,68}}    & 0,66          \\ \hline
        \end{tabular}
\end{table}

\clearpage

% --- PCA 1o exp --- %
% \begin{landscape}
\begin{figure}[h]
    \centering
    \includegraphics[width=1.5\textwidth]{img/pca64.png}
    \caption{\label{fig:pca64}PCA com 2 componentes aplicado ao resultado do \textit{encoding} do primeiro experimento}
\end{figure}
% \end{landscape}

% \begin{landscape}
\begin{figure}[h]
    \centering
    \includegraphics[width=1.5\textwidth]{img/pca64-2.png}
    \caption{\label{fig:pca64-2}Agrupamento em 2 classes para PCA com 2 componentes aplicado ao resultado do \textit{encoding} do primeiro experimento}
\end{figure}
% \end{landscape}

\clearpage

\subsection{Segundo experimento: 128 \textit{MFCC}s}

A Figura \ref{fig:ae128} apresenta a arquitetura do modelo Autoencoder, onde podemos observar que os vetores do espaço latente, portanto, o vetor comprimido para ser posteriormente reconstruído, terão $64$ posições.

Para este modelo, o valor final da \textit{Loss} média ao aplicá-lo ao seu conjunto de teste foi de $0,84$. A Figura \ref{fig:predae128} ilustra uma predição feita pelo Autoencoder: A linha azul representa o dado original (\textit{input}); a linha tracejada laranja representa a predição (\textit{output}) daquele registro; e a linha verde foi calculada tomando a diferença entre os coeficientes do \textit{input} e do \textit{output}.

A Figura \ref{fig:clf128} apresenta a arquitetura do modelo de classificação de intensidade, enquanto a Tabela \ref{table:metricasclf128} apresenta métricas de desempenho para sua aplicação no conjunto de teste. Podemos observar que o melhor resultado para do \textit{F1-Score} deu-se para a intensidade Pico, com o valor de $0,64$, ficando abaixo de $0,5$ apenas para a classe Forte.

Aplicamos o classificador aos dados oriundos do $X_{VERBO}$ e anotamos o resultado. Então efetuamos um \acrshort{PCA} com 2 componentes ao resultado do \textit{encoding} de $X$ (Figura \ref{fig:pca128}), onde as cores representam a classe - original para os dados do $VIVAE$ e predição do classificador para os dados do $VERBO$ - e podemos observar a formação de 2 grandes \textit{clusters}, um a esquerda e outro a direita.

Percebemos através dos \textit{labels}, que estes \textit{clusters} conseguiram distinguir bem as duas bases de dados, bem como um movimento descendente de aumento da intensidade da emoção em ambos os \textit{clusters}. Temos uma visualização mais simples na Figura \ref{fig:pca128-2}, onde agrupamos as classes Fraca e Moderada em uma nova classe denominada Baixa e as classes Forte e Pico em uma classe denominada Alta.

% --- PRED AE 128 --- %
    \begin{figure}%[t]
    \centering
    \includegraphics[width=0.6\textwidth]{img/_predae128.png}
    \caption{\label{fig:predae128}Exemplo de predição (\textit{encoding e decoding}) do \textit{Autoencoder} para 128 \textit{MFCC}s}
\end{figure}

% --- CLF 128 Metrics --- %
\begin{table}%[h]
    \centering
    \begin{tabular}{l|ccc|}
        \cline{2-4}
                                          & \multicolumn{3}{c|}{Métricas}                                                               \\ \hline
        \multicolumn{1}{|l|}{Intensidade} & \multicolumn{1}{c|}{Precision}      & \multicolumn{1}{c|}{Recall}           & F1-Score      \\ \hline
        \multicolumn{1}{|l|}{Fraca}       & \multicolumn{1}{c|}{\textbf{0,62}}  & \multicolumn{1}{c|}{0,55}             & 0,58          \\ \hline
        \multicolumn{1}{|l|}{Moderada}    & \multicolumn{1}{c|}{0,55}           & \multicolumn{1}{c|}{0,48}             & 0,51          \\ \hline
        \multicolumn{1}{|l|}{Forte}       & \multicolumn{1}{c|}{0,49}           & \multicolumn{1}{c|}{0,50}              & 0,49          \\ \hline
        \multicolumn{1}{|l|}{Pico}        & \multicolumn{1}{c|}{0,57}           & \multicolumn{1}{c|}{\textbf{0,73}}    & \textbf{0,64} \\ \hline
    \end{tabular}
        \caption{\label{table:metricasclf128}Métricas de avaliação para o classificador do experimento para 128 \textit{MFCCs}}
\end{table}

% --- PCA 2o exp --- %
\begin{figure}%[h]
    \centering
    \includegraphics[angle=90, width=0.8\textwidth]{img/pca128.png}
    \caption{\label{fig:pca128}PCA com 2 componentes aplicado ao resultado do \textit{encoding} do segundo experimento}
\end{figure}

\begin{figure}%[h]
    \centering
    \includegraphics[angle=90, width=0.8\textwidth]{img/pca128-2.png}
    \caption{\label{fig:pca128-2}Agrupamento em 2 classes para PCA com 2 componentes aplicado ao resultado do \textit{encoding} do segundo experimento}
\end{figure}



\section{Discussão dos resultados}

Nesta seção iremos iniciar uma discussão sobre os resultados alcançados e apresentaremos as dificuldades encontradas, além das limitações gerais do projeto.\\

Com base nos resultados da Tabela \ref{table:resultexp} verificamos que o primeiro experimento obteve desempenho superior no que tange a métrica selecionada para o modelo de classificação de intensidade, tendo um \textit{F1-Score} superior aos do primeiro experimento em três das quatro classes.

Em ambos experimentos, a classe com pior resultado do classificador - de acordo a métrica definida na metodologia - foi semelhante, a classe Forte. E as classes com melhor desempenho são Fraca e Pico de intensidade, respectivamente.

Dadas as Figuras \ref{fig:pca64} e \ref{fig:pca128}, conseguimos observar que as classes Fraca e Pico ocupam os extremos dos \textit{clusters}, o que pode tornar sua separação mais fácil frente às amostras das classes Moderada e Forte, as quais podemos notar mais amalgamadas na região central. Esta distribuição parece estar de acordo com nossos resultados, uma vez que os dois melhores desempenhos - em ambos os cenários - foram das classes Fraca e Pico, e os piores das classes Moderada e Forte.

Na Tabela \ref{table:compexp} vemos um ganho de desempenho superior a 7 vezes para o valor da \textit{Loss} no segundo experimento, no qual utilizamos um número maior de \acrshort{MFCC}s. Embora a \textit{Loss} tenha apresentado uma queda significativa no segundo experimento, o que significa que o \textit{Autoencoder} apresenta um desempenho melhor para reproduzir o dado de entrada, preservando melhor as características e atributos do dado, essa melhora no desempenho de uma função identidade se mostra contraditória frente o desempenho do segundo classificador. Então, podemos supor que uma quantidade estritamente maior de \acrshort{MFCC}s colaborou para a reconstrução da amostra enquanto não demonstrou ganhos semelhantes na sua utilização para classificar a intensidade da emoção.

\begin{table}[]
    \centering
    \begin{tabular}{|l|l|l|}
    \hline
        Intensidade & F1-Score para 64 MFCCs & F1-Score para 128 MFCCs \\ \hline
        Fraca & \textbf{0,68} & 0,58 \\ \hline
        Moderada & \textbf{0,53} & 0,51 \\ \hline
        Forte & 0,48 & \textbf{0,49} \\ \hline
        Pico & \textbf{0,66} & 0,64 \\ \hline
    \end{tabular}
     \caption{\label{table:resultexp}Comparativo de \textit{F1-Score} entre os experimentos}
\end{table}

\begin{table}[]
    \centering
    \begin{tabular}{|l|l|l|}
    \hline
        Resultado & MFCCs & ~ \\ \hline
        ~ & 64 & 128 \\ \hline
        Maior F1-Score & \textbf{0,68} & 0,64 \\ \hline
        Classe com maior F1-Score & Fraca & Pico \\ \hline
        Menor F1-Score & 0,48 & \textbf{0,49} \\ \hline
        Classe com menor F1-Score & Forte & Forte \\ \hline
    \end{tabular}
    \caption{\label{table:compexp}Comparativo de atributos de desempenho dos experimentos}
\end{table}

%Embora nossa massa de dados seja composta por poucos registros - quando comparamos com outras bases de dados encontradas na literatura -
De posse dos resultados dos experimentos, conseguimos observar que os vetores do espaço latente dos \textit{Autoencoders}, mesmo com uma redução de dimensionalidade à metade, aparentam manter características relativas a intensidade da emoção presentes ambos experimentos.

Quando agrupamos as classes Fraca e Moderada no cojunto denominado Baixo e as classes Forte e Pico no conjunto denominado Alto, conseguimos observar nas Figuras \ref{fig:pca64-2} e \ref{fig:pca128-2} que mesmo com o desempenho superior do classificador do primeiro experimento, ainda há uma correspondência entre as intensidades dos registros. Uma vez que os \textit{labels} de $X_{VIVAE}$ são originais, observamos que as classes atribuídas aos dados de $X_{VERBO}$ parecem ser condizentes, vide o comportamento descendente da aumento da intensidade. Assim, o classificador aprendeu com $X_{VIVAE}$ a classificar a intensidade e aplicou essa lógica aos dados em $X_{VERBO}$.

Dados os rótulos corretos para os dados do VIVAE, essa visualização fornece outro \textit{insight} de como seríamos capazes de traçar uma linha em $Component\ 2 = 0$ de forma que os dados para ambos os experimentos possam ser divididos em duas categorias macro: Baixa (Fraco e Moderada) e Alta (Forte e Pico). Assim, para ambos os experimentos, dado um ponto de dados $x_i = (i_{component_1}, i_{component_2})$, se $i_{component_2} >= 0$ ele pertenceria à classe Baixa, enquanto se $i_{component_2} < 0$ ele pertenceria à classe Alta. Essa interpretação ingênua levanta a observação de que o Componente 2 seria responsável pela intensidade de uma dada declaração de maneira quase que exclusiva.
