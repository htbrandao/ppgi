% \chapter{Conclusões}\label{Cap:Conclusoes}

Neste capítulo serão abordadas as considerações finais sobre o projeto. Elicitaremos quais objetivos foram alcançados e contribuições que o projeto traz. Também serão levantadas as dificuldades encontradas e as limitações desta pesquisa. Por fim, serão elucidados os possíveis trabalhos futuros.\\

\textcolor{blue}{Neste trabalho nós contextualizamos um problema de pesquisa, bem delimitado, e vimos um breve panorama da literatura relacionada. Em seguida, formalizamos a metodologia e os componentes do trabalho proposto para, finalmente, realizar os experimentos com base nessa proposta. Buscando realizar uma tarefa de aprendizado de máquina para inferir a intensidade das emoções na voz no nosso idioma nativo (Português), até então não encontrada na literatura.}

\textcolor{blue}{Os resultados indicam que parece ser possível inferir a intensidade. Porém, o conjunto de dados ainda é bastante escasso. Também não temos conhecimento de uma base de dados em português que apresente tanto emoções quanto suas intensidades. Embora o português seja uma língua falada pela sexta\footnote{Disponível em \url{https://brasilescola.uol.com.br/geografia/populacao-mundial.htm}} maior população e pela nona\footnote{Disponível em \url {https://www.gov.br/funag/pt-br/ipri/publicacoes/estatisticas/as-15-maiores-economias-do-mundo}} maior economia do mundo, quando comparamos o VERBO com conjuntos de dados como o AudioSet, percebemos a enorme distância tanto em número de amostras ($\approx 2.000.000$) como em duração média ($\approx 10s$). É de esperar que uma base de dados mais robusta melhore o desempenho da investigação.}

% Para trabalhos futuros, destacamos a necessidade de conseguir uma massa de dados com mais registros, o que é uma queixa recorrente na literatura correlata. Pretendemos experimentar outras arquiteturas de redes neurais, como \textit{CNNs} e \textit{RNNs} para extração de características e classificação de intensidade, e também desejamos utilizar outros tipos de características da fala, como \textit{features} prosódicas.

\textcolor{blue}{Para trabalhos futuros, pretendemos implementar uma Rede Neural Recorrente (\textit{RNN}) para avaliar os dados ao longo do eixo do tempo tentando melhorar o desempenho do modelo de classificação. Pretendemos também realizar análises exploratórias para entender se o modelo está apresentando algum tipo de viés (como a classe Forte tendo o pior desempenho em ambos experimentos), e em caso afirmativo, entender como mitigá-lo. Outra abordagem seria dar um passo atrás e enfrentar um grande desafio em nossa tarefa e criar nosso próprio conjunto de dados com emoções e intensidade, sendo o primeiro em nossa língua nativa (Português do Brasil).}

\section{Principais Contribuições}

...

...

\section{Trabalhos Futuros}

...

...

