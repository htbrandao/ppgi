Speech is often our first form of communication and expression of emotions. Speech Emotion Recognition is a complex problem, as emotional expression depends on the spoken language, dialect, accent and cultural background of individuals. The intensity of this emotion can affect our perception of emotions and lead us to interpret information inappropriately. Even with the prospect of applicability in various areas such as patient monitoring, security, commercial systems and entertainment, there seems to be a lack of work addressing the classification of emotion intensity in Portuguese. Regardless of training as mental health professionals, it is natural that we can assign some kind of metric to compare two instances of an emotion that we have experienced. Thus, we are able to experience and compare different intensities for the same emotion. Therefore, we are able to identify emotions, quantify their intensity and calculate a distance to make this comparison. Works using Machine Learning applied on emotion recognition from speech have been published - at least - since the beginning of the 1990s. Having found no occurrence in the literature, this work carried out a Machine Learning task using both Machine Learning and Deep Learning to infer the intensity of emotions in the voice in Portuguese.