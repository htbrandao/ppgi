A fala costuma ser a nossa primeira forma de comunicação e de expressão de emoções. O Reconhecimento de Emoção na Fala \acrfull{SER} é um problema complexo, pois a expressão emocional depende da linguagem falada, dialeto, sotaque e histórico cultural dos indivíduos. A intensidade dessa emoção pode afetar nossa percepção das emoções e nos induzir a interpretar a informação de forma inadequada. Mesmo com perspectiva de aplicabilidade em diversas áreas como monitoramento de pacientes, segurança, sistemas comerciais e entretenimento, parece haver uma ausência de trabalhos abordando a classificação da intensidade da emoção em português. Independente de formação enquanto profissionais de saúde mental, é natural que consigamos atribuir alguma espécie de métrica para comparar duas instâncias de uma mesma emoção que tenhamos sentido. Assim, conseguimos experienciar e comparar intensidades distintas para uma mesma emoção. Portanto, somos capazes de identificar emoções, quantificar sua intensidade e calcular uma distância para efetuar essa comparação. Trabalhos de Aprendizado de Máquina \acrfull{ML} aplicados ao reconhecimento de emoções na fala vêm sendo publicados - ao menos - desde o início da década de 90. Não tendo encontrado ocorrência na literatura, este trabalho realizou uma tarefa de \acrshort{ML} utilizando tanto Aprendizado de Máquina quanto Aprendizado Profundo \acrfull{DL} para inferir a intensidade das emoções na voz em português