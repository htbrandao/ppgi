% \subsection{Primeiro experimento: 64 \textit{MFCC}s}

A Figura \ref{fig:ae64} apresenta a arquitetura do modelo Autoencoder, onde podemos observar que os vetores do espaço latente, portanto, o vetor comprimido para ser posteriormente reconstruído, terão $32$ posições.

Para este modelo, o valor final da \textit{Loss} média ao aplicá-lo ao seu conjunto de teste foi de $6,40$. A Figura \ref{fig:predae64} ilustra uma predição feita pelo Autoencoder: A linha azul representa o dado original (\textit{input});a linha tracejada laranja representa a predição (\textit{output}) daquele registro; e a linha verde foi calculada tomando a diferença entre os coeficientes do \textit{input} e do \textit{output}.

% --- PRED AE 64 --- %
\begin{figure}[h]
    \centering
    \includegraphics[width=0.6\textwidth]{img/_predae64.png}
    \caption{\label{fig:predae64}Predição (\textit{encoding e decoding}) do \textit{Autoencoder} para 64 \textit{MFCC}s}
\end{figure}

A Figura \ref{fig:clf64} apresenta a arquitetura do modelo de classificação de intensidade, enquanto a Tabela \ref{table:metricasclf64} apresenta métricas de desempenho para sua aplicação no conjunto de teste. Podemos observar que o melhor resultado para o \textit{F1-Score} deu-se para a intensidade Fraca, com o valor de $0,68$, e ficando abaixo de $0,5$ apenas para a classe Forte.

Aplicamos o classificador aos dados oriundos do $X_{VERBO}$ e anotamos o resultado. Então efetuamos um \textit{PCA} com 2 componentes ao resultado do \textit{encoding} de $X$ (Figura \ref{fig:pca64}), onde as cores representam a classe - original para os dados do $VIVAE$ e predição do classificador para os dados do $VERBO$ - e podemos observar a formação de 2 grandes \textit{clusters}, um a esquerda e outro a direita.

Percebemos através dos \textit{labels}, que estes \textit{clusters} conseguiram distinguir bem as duas bases de dados, bem como um movimento descendente de aumento da intensidade da emoção em ambos os \textit{clusters}. Temos uma visualização mais simples na Figura \ref{fig:pca64-2}, onde agrupamos as classes Fraca e Moderada em uma nova classe denominada Baixa e as classes Forte e Pico em uma classe denominada Alta.

% --- CLF 64 Metrics --- %
\begin{table}[h]\caption{\label{table:metricasclf64}Métricas de avaliação para o classificador do experimento para 64 \textit{MFCCs}}
    \centering
    \begin{tabular}{l|ccc|}
        \cline{2-4}
                                      & \multicolumn{3}{c|}{Métricas}                                                               \\ \hline
            \multicolumn{1}{|l|}{Intensidade} & \multicolumn{1}{c|}{Precision}      & \multicolumn{1}{c|}{Recall}           & F1-Score      \\ \hline
            \multicolumn{1}{|l|}{Fraca}       & \multicolumn{1}{c|}{\textbf{0,73}}  & \multicolumn{1}{c|}{0,64}             & \textbf{0,68} \\ \hline
            \multicolumn{1}{|l|}{Moderada}    & \multicolumn{1}{c|}{0,53}           & \multicolumn{1}{c|}{0,52}             & 0,53          \\ \hline
            \multicolumn{1}{|l|}{Forte}       & \multicolumn{1}{c|}{0,46}           & \multicolumn{1}{c|}{0,50}              & 0,48          \\ \hline
            \multicolumn{1}{|l|}{Pico}        & \multicolumn{1}{c|}{0,64}           & \multicolumn{1}{c|}{\textbf{0,68}}    & 0,66          \\ \hline
        \end{tabular}
\end{table}

\clearpage

% --- PCA 1o exp --- %
% \begin{landscape}
\begin{figure}[h]
    \centering
    \includegraphics[width=1.5\textwidth]{img/pca64.png}
    \caption{\label{fig:pca64}PCA com 2 componentes aplicado ao resultado do \textit{encoding} do primeiro experimento}
\end{figure}
% \end{landscape}

% \begin{landscape}
\begin{figure}[h]
    \centering
    \includegraphics[width=1.5\textwidth]{img/pca64-2.png}
    \caption{\label{fig:pca64-2}Agrupamento em 2 classes para PCA com 2 componentes aplicado ao resultado do \textit{encoding} do primeiro experimento}
\end{figure}
% \end{landscape}

\clearpage
