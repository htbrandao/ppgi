\chapter{Introdução}\label{Cap:Introdução}

% ==========================================================================================
\section{Contextualização e Problema}

A fala costuma ser nossa primeira forma de comunicação e de expressão de emoções \cite{1.5}. Desde a infância, antes mesmo de utilizarmos nosso idioma, já expressamos emoções através de sons não verbais mas que possuem significado para o emissor - havendo, inclusive, estudos investigando a emoção de bebês a partir do seu choro \cite{0} - e, ao longo da vida, continuamos a expressar emoções dessa maneira. Em um mundo moderno, estamos interagindo cada vez mais com, e através de, ferramentas tecnológicas (e.g.: assistentes virtuais como Alexa e Siri). Desse modo, a classificação das emoções e de sua intensidade toma um papel importante \cite{3} no desenvolvimento de ciência e tecnologia, uma vez que mensagem transmitida pode ter sua semântica alterada pelas emoções \cite{39}.

A emoção é um estado psicológico relacionado com o sistema sensorial, provocado por alterações hormonais que podem estar relacionadas a observações, sentimentos, iterações sociais ou algum nível de satisfação ou frustração, que causam uma alteração, distinguível, na fala \cite{8}. A intensidade dessa emoção pode afetar nossa percepção das emoções \cite{18.46} e nos induzir a interpretar de forma inadequada. A análise de emoções na voz se tornou um área de pesquisa proeminente, muito graças ao aumento da capacidade computacional e a eficiência de algorítmos \cite{38} \cite{20}.

O Reconhecimento de Emoção na Fala (\textit{Speech Emotion Recognition}, \textit{SER}) é um problema complexo, pois a expressão emocional depende da linguagem falada, dialeto, sotaque e histórico cultural dos indivíduos \cite{6}. O reconhecimento e avaliação de emoções apresenta dificuldades por sua natureza interdisciplinar: o reconhecimento de emoções e a avaliação de força são objeto das ciências da psicologia, a aferição e avaliação de dados fisiológicos estão relacionadas às ciências médicas, e a análise e solução de dados de sensores é o objeto da mecatrônica \cite{17}. No âmbito das Ciências da Computação, \textit{SER} é uma área de pesquisa ativa, com publicações datando desde o final do século XX \cite{12.27}.

Pesquisando o estado da arte da literatura relacionada, encontramos diversos trabalhos lidando com \textit{SER}, como \cite{11} que investiga características rítmicas, \cite{34} que emprega uma abordagem (Não Supervisionada) menos comum na literatura, para tentar capturar estruturas intrínsecas dos dados, \cite{32.95} que utiliza um mecanismo de atenção\footnote{Introduzido em 2014 por Dzmitry Bahdanau, Kyunghyun Cho e Yoshua Bengio. Disponível em \url{https://arxiv.org/abs/1409.0473}}. Entretanto, encontrarmos menos trabalhos abordando a intensidade dessas emoções. Com esse escopo, podemos citar \cite{14}, \cite{15} e \cite{28}, embora utilizem dados textuais, e \cite{3}, \cite{18} e \cite{20} que utilizam voz. Nenhum destes trabalhos utiliza dados em português.

Realizando \textit{SER} em nosso idioma nativo, temos \cite{12} que utiliza modelos especialistas para cada emoção abordada e \cite{21} que utiliza modelos tradicionais, como Máquinas de Vetores de Suporte. Ambos utilizam uma abordagem chamada de Supervisionada. Nenhum destes trabalhos tratam da itensidade das emoções.

Portanto, haveria a possibilidade deste trabalho colaborar para o estado da arte de \textit{SER} em Português envolvendo a inferência da intensidade das emoções.

% ==========================================================================================
\section{Objetivos}

A estimativa de intensidade na emoção tem diversas aplicações potenciais para interações humano-máquina, como monitoramento de pacientes \cite{1}, segurança \cite{4}, comerciais\footnote{Behavioral Signals. Exemplo de empresa que utiliza atividades correlatas a este trabalho para fins comerciais. Disponível em \url{https://behavioralsignals.com/}} e de entretenimento. Este trabalho pretende criar uma arquitetura de aprendizado de máquina para classificaçar a intensidade de emoções no idioma Português Brasileiro. Até então não encontrada na literatura.

\subsection{Objetivos específicos}

% Para estes objetivos, compreende-se como fala vocalizações verbais no idioma português brasileiro.

\begin{enumerate}
    \item Criar uma arquitetura de aprendizado de máquina para classificação de intensidade da emoção na fala em português brasileiro;
    \item Validar a solução criada no item anterior;
    \item Confrontar os resultados obtidos com os possíveis modelos do estado da arte, e caso não seja possível, criar modelos simples (\textit{vanilla}) e comparar os desempenhos;
    \item Especificar o conjunto de técnicas que sejam adequados para realizar a tarefa a que esse trabalho se propõe.
\end{enumerate}

% ==========================================================================================
\section{Estrutura deste Trabalho}

Esta proposta de trabalho está organizada como se segue. Os demais capítulos, enumerados de 2 a 5, apresentam, respectivamente: Fundamentação Teórica, Trabalhos Relacionados, Projeto de Pesquisa, Cronograma de Trabalho. No capítulo 2 é feita uma fundamentação para oferecer os conceitos necessários para a compreensão do texto. No capítulo 3, uma discussão e comparativo a respeito da literatura relacionada a esta pesquisa. No capítulo 4 o projeto proposto é apresentado e definido. No capítulo 5 é apresentado o cronograma de atividades desta pesquisa científica.
