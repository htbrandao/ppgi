% \chapter{Introdução}\label{Cap:Introdução}

% ==========================================================================================
\section{Contextualização}

A fala costuma ser nossa primeira forma de comunicação e de expressão de emoções~\cite{1.5}. Desde a infância, antes mesmo de utilizarmos nosso idioma, expressamos emoções através de sons não verbais que possuem significado para o emissor. Havendo, inclusive, estudos investigando a emoção de bebês por meio do choro~\cite{0}.

Continuamos a expressar emoções dessa maneira ao longo da vida. Em um mundo moderno, estamos interagindo cada vez mais com, e através de, ferramentas tecnológicas (\textit{e.g.}: assistentes virtuais como Alexa e Siri).

Nossa emoção é um estado psicológico relacionado com o sistema sensorial, provocado por alterações hormonais que podem estar relacionadas a observações, sentimentos, interações sociais ou algum nível de satisfação ou frustração, e que causam uma alteração, distinguível, na fala~\cite{8}.

A intensidade dessa emoção pode afetar nossa percepção das emoções~\cite{18.46} e nos induzir a interpretá-las de forma inadequada.

A análise de emoções na voz se tornou uma área de pesquisa proeminente, muito graças ao aumento da capacidade computacional e a eficiência de algoritmos~\cite{38}~\cite{20}. Desse modo, a classificação de emoções e de sua intensidade toma um papel importante~\cite{3} no desenvolvimento de ciência e tecnologia, uma vez que a mensagem transmitida pode ter sua semântica alterada pelas emoções~\cite{39}.

O Reconhecimento de Emoção na Fala é um problema complexo, pois a expressão emocional depende da linguagem falada, dialeto, sotaque e histórico cultural dos indivíduos~\cite{6}. O reconhecimento e avaliação de emoções apresenta dificuldades por sua natureza interdisciplinar: O reconhecimento de emoções e a avaliação de intensidade são objeto das ciências da Psicologia, a aferição e avaliação de dados fisiológicos estão relacionadas às ciências médicas, e a análise e solução de dados de sensores é objeto da mecatrônica~\cite{17}.

% ==========================================================================================
\section{Objetivos}

No âmbito das Ciências da Computação, \acrshort{SER} é uma área de pesquisa ativa, com publicações datando desde o final do século XX~\cite{12.27}.

Pesquisando o estado da arte da literatura relacionada, encontramos trabalhos lidando com \acrshort{SER}, como~\cite{11} que investiga características rítmicas~\cite{34} para tentar capturar estruturas intrínsecas dos dados e~\cite{32.95} que utiliza um mecanismo de atenção\footnote{Introduzido em 2014 por Dzmitry Bahdanau, Kyunghyun Cho e Yoshua Bengio. Disponível em \url{https://arxiv.org/abs/1409.0473}}. Encontrarmos trabalhos abordando a intensidade dessas emoções, dos quais podemos citar~\cite{14},~\cite{15} e~\cite{28}, embora utilizem dados textuais, e~\cite{3},~\cite{18} e~\cite{20} que utilizam voz. Mas nenhum destes trabalhos utiliza dados em português.

Pesquisando \acrshort{SER} em nosso idioma nativo, podemos citar~\cite{12} que utiliza modelos especialistas para cada emoção abordada e~\cite{21} que utiliza modelos tradicionais, como Máquinas de Vetores de Suporte \acrfull{SVM}, mas nenhum destes trabalhos tratam da intensidade das emoções.

Portanto, haveria a possibilidade desta dissertação colaborar para o estado da arte de \acrshort{SER} em Português através da inferência da intensidade das emoções.

Uma vez que a estimativa de intensidade na emoção tem diversas aplicações potenciais para interações humano máquina, como: Monitoração de pacientes~\cite{1}, segurança~\cite{4}, fins comerciais\footnote{Behavioral Signals. Exemplo de empresa que utiliza atividades correlatas a esta para fins comerciais. Disponível em \url{https://behavioralsignals.com/}} e de entretenimento (\textit{e.g.}: Assistentes virtuais).

Nesta dissertação, realizamos uma tarefa para classificar a intensidade da emoção na voz em Português Brasileiro. Para isso, foram utilizadas duas bases de dados composta por vocalizações verbais e não verbais, nas quais realizamos extração de características para posteriormente treinarmos dois modelos de \acrshort{DL}: O primeiro para extrair características dos dados e o segundo para efetuar a classificação da intensidade.


% ==========================================================================================
% \section{Objetivos}

% A estimativa de intensidade na emoção tem diversas aplicações potenciais para interações humano máquina, como monitoramento de pacientes~\cite{1}, segurança~\cite{4}, comerciais\footnote{Behavioral Signals. Exemplo de empresa que utiliza atividades correlatas a este trabalho para fins comerciais. Disponível em \url{https://behavioralsignals.com/}} e de entretenimento. Este trabalho pretende desenvolver uma arquitetura de aprendizado de máquina para classificar a intensidade de emoções em falas do idioma Português Brasileiro.

% \subsection{Objetivos específicos}

% % Para estes objetivos, compreende-se como fala vocalizações verbais no idioma português brasileiro.

% \begin{enumerate}
%     \item Desenvolver uma arquitetura de aprendizado de máquina para classificação de intensidade da emoção na fala em português brasileiro;
%     \item Utilizar um algoritmo de aprendizado de máquina para validar a solução criada no item anterior;
%     % \item Confrontar os resultados obtidos com os possíveis modelos do estado da arte, e caso não seja possível, criar modelos simples (\textit{vanilla}) para comparar a métrica selecionada como indicador de desempenho;
%     \item Especificar um conjunto de técnicas que sejam adequados para realizar a tarefa a que esse trabalho se propõe.
% \end{enumerate}

% ==========================================================================================
\section{Estrutura deste Trabalho}

Esta dissertação está organizada como se segue. Os demais capítulos, enumerados de 2 a 6, apresentam, respectivamente: Fundamentação Teórica, Trabalhos Relacionados, Pesquisa, Resultados, e Conclusões. No capítulo 2 é feita uma fundamentação para oferecer os conceitos necessários para a compreensão do texto; no capítulo 3, são apresentadas uma discussão e comparação a respeito da literatura relacionada; no capítulo 4 apresentamos a estratégia empregada nesta pesquisa; no capítulo 5 iremos detalhar a implementação, apresentar e discutir os resultados; e, por fim, no capítulo 6, apresentamos as conclusões desta dissertação.
