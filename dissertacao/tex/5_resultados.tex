% \chapter{Resultados}\label{Cap:Resultados}

Neste capítulo serão apresentados os detalhes da implementação, resultados, métricas de desempenho e a metodologia utilizada na avaliação deste trabalho.\\

\section{Classificação de Intensidade das Emoções na Fala em Português Brasileiro por meio de Deep Learning}\label{sec:implementacao}

Esta seção detalha nossa implementação para o efetuar o reconhecimento da intensidade emoções na fala em nosso idioma nativo. Utilizando aprendizagem supervisionadas e não supervisionada para uma tarefa de reconhecimento da intensidade da emoção expressa vocalmente.

Para tanto, em virtude do desafio da demanda, ratificado pela escassez de dados, nos aproveitamos da Fusão de Dados para criar, primeiramente, uma solução não supervisionada que consiga extrair características que sejam representativas o suficiente para que, mesmo com a dimensionalidade reduzida em comparação a original, consigamos reconstruir o dado de entrada; e posteriomente utilizar esses dados comprimidos, constituído de atributos suficientemente relevantes, para desenvolver um modelo supervisionado de inferência da intensidade.

Definida a arquitetura composta por um Autoencoder (\ref{sec:bravo_ae}) e um Classificador (\ref{sec:bravo_clf}), tomamos apenas as emoções comuns (\ref{sec:bravo_dados}) entre as duas bases de dados (alegria, medo, raiva e surpresa), totalizando 1364 registros, onde 51\% apresentam a o \textit{label} relativo a intensidade. %A Figura \ref{fig:design} apresenta o \textit{design} do \textit{BRAVO}.

% \begin{figure}[h]
%     \centering
%     \includegraphics[width=1.0\textwidth]{img/design.PNG}
%     \caption{\label{fig:design}\textcolor{blue}{\textit{Design} da proposta para o \textit{BRAVO}}}
% \end{figure}

Foram realizados dois cenários experimentais, utilizando 64 e 128 \textit{MFCCs}, respectivamente. Em ambos,  o \textit{Autoencoder} foi treinado com a totalidade desses registros. Em seguida treinamos uma rede neural densa multicamada para classificar a intensidade, treinada e validada apenas na porção dos dados oriunda do $VIVAE$, uma vez quer o $VERBO$ não tem correspondência com o contradomínio ($Z$) das intensidades.

Há de se observar que não há \textit{label} correspondente às intensidades par $X_{VERBO}$. Então precisamos investigar se há algum sentido nos resultados quando aplicarmos a classificação a esses dados, que até então ainda não foram vistos pelo classificador. Assim, faz-se necessária uma forma de analisar os registos de $X_{VIVAE}$ e$X_{VERBO}$ quanto às intensidades e a predição dessas intensidades, respectivamente. Podemos utilizar o \textit{PCA} (\textit{Principal Component Analysis}) para reduzir a dimensionalidade dos registros e observar o comportamento da classificação. A utilização \textit{PCA} encontra registros na literatura, tanto de \textit{ML} e \textit{DL} aplicados a voz e emoções. Temos ~\cite{pca1} que utilizou o PCA para gerar features de um modelo, observando que modelos treinados com as features obtidas a partir do PCA mantiveram sua performance quando comparados aos treinados com outros atributos; ~\cite{pca2} que apesar de fazer uma \textit{feature selection} prévia, optou por utilizar o \textit{PCA} para reduzir em quase três vezes a dimensionalidade do dado para conseguir uma visualização bidimensional dos registros; ~\cite{pca3} buscou desenvolver um sistema de reconhecimento de emoções para áudios com ruídos, utilizando dados $64$-dimensionais que seriam reduzidos para um espaço $6$-dimensional utilizando e comparando técnicas distintas, sendo uma delas o \textit{PCA}; e ~\cite{pca5} que aproxima o grau de \textit{encoding} de um \textit{Autoencoder} linear - em seu espaço latente $n$-dimensional - ao de um PCA com $n$ componentes. Assim, vemos o \textit{PCA} ser utilizado tanto para permitir a visualização de dados através da redução de dimensionalidade quanto para gerar as \textit{features} utilizadas em modelos de reconhecimento de emoções. O que nos diz que o \textit{PCA} seria uma solução adequada para reduzir a dimensionalidade dos dados enquanto preserva características relevantes das amostras.

% ------------------------------------------------------------------------------------------
\subsection{Autoencoder}

A estrutura dos Autoencoders é análoga para ambos os experimentos. A Figura \ref{fig:ae64} representa a arquitetura do modelo para 64 \textit{MFCC}s enquanto a \ref{fig:ae128} representa a arquitetura do modelo para 128 \textit{MFCC}s. Ambos seguem uma estrutura composta por três camadas, a serem detalhadas abaixo: Entrada, \textit{encoder} e \textit{decoder}. O treinamento foi realizado utilizando \textit{MSE} como função de \textit{Loss} e \textit{ADAM} como função de otimização. Para efeitos de legibilidade, vamos definir $dim_{MFCCs}$ como a quantidade de \textit{MFCC}s utilizado em cada experimento.

\begin{enumerate}
    \item Entrada: Responsável por receber os dados com dimensão igual $(1, dim_{MFCCs})$.
    \item \textit{Encoder}: Responsável por realizar a redução de dimensionalidade do dado oriundo da camada de entrada. Na camada de \textit{encoder}, o dado entra com dimensão $(1, dim_{MFCCs})$ e é comprimido para uma dimensionalidade $(1, dim_{MFCCs}/2)$. Assumimos que buscamos preservar apenas os atributos com maior relevância, então utilizamos uma função de ativação do tipo Unidade Linear Retificada (\textit{Linear Rectified Unit}, \textit{ReLU}).
    \item \textit{Decoder}: Responsável por receber os dados com dimensão reduzida e expândi-los para a dimensão de entrada $(1, dim_{MFCCs})$, tentando reproduzir os valores originais. Para isso, esta camada utilizou uma função de ativação do tipo Linear, que reproduzirá valores contítnuos oriundos das operações matriciais desta camada, não limitados ao intervalo $(0, max(0, x)$ como é o caso da \textit{ReLU}.
\end{enumerate}

% ---- Arquiteturas AEs ---- %
\begin{figure}[h]
    \centering
    \begin{minipage}[b]{0.4\linewidth}
        \centering
        \includegraphics[width=\linewidth]{img/ae64.png}
        \caption{\label{fig:ae64}Arquitetura do Autoencoder para 64 \textit{MFCCS}}
    \end{minipage}
    % \hfill
    \begin{minipage}[b]{0.4\linewidth}
        \centering
        \includegraphics[width=\linewidth]{img/ae128.png}
        \caption{\label{fig:ae128}Arquitetura do Autoencoder para 128 \textit{MFCCS}}
    \end{minipage}
\end{figure}

% ------------------------------------------------------------------------------------------
\subsection{Classificador}

A estrutura dos Classificadores é análoga para ambos os experimentos. A Figura \ref{fig:clf64} representa a arquitetura do modelo para 64 \textit{MFCC}s enquanto a \ref{fig:clf128} representa a arquitetura do modelo para 128 \textit{MFCC}s. Ambos seguem uma estrutura composta por cinco camadas, a serem detalhadas abaixo: Entrada, três camadas densas intermediárias e uma camada densa de saída. O treinamento foi realizado utilizando \textit{Categorical Cross Entropy} como função de \textit{Loss} e \textit{ADAM} como função de otimização. Uma vez que o \textit{AE} realiza a função de reduzir a dimensionalidade dos dados que serão utilizados como \textit{input} para os classificadores, para efeitos de legibilidade vamos definir $dim_{encode}$ como a dimensão do dado comprimido obtido nos \textit{Autoencoders} da etapa anterior, sendo estas 32 e 64, respectivamente.

\begin{enumerate}
    \item Entrada: Responsável por receber os dados com dimensão igual $(1, dim_{encode})$.
    \item Densas intermediárias: As camadas posteriores (\textit{dense}, \textit{dense\_1} e \textit{dense\_2}) são camadas totalmente conectadas que utilizam a função de ativação \textit{ReLU} e cujo resultado é um vetor com dimensão $(1, 8)$.
    \item Densa de saída: A camada \textit{dense\_3} é responsável por efetuar a ativação final da rede neural e entrega um vetor com dimensão $(1, 4)$. Utiliza uma função de ativação \textit{Softmax}, cujo resultado será interpretado para classificar a intensidade da emoção em uma de quatro categorias: Fraca, moderada, forte ou pico de intensidade.
\end{enumerate}

% ---- Arquiteturas CLFs ---- %
\begin{figure}[]
    \centering
    \begin{minipage}[b]{0.45\linewidth}
        \centering
        \includegraphics[width=\linewidth]{img/clf64.png}
        \caption{\label{fig:clf64}Arquitetura do classificador do experimento com 64 \textit{MFCCs}}
    \end{minipage}
    % \hfill
    \begin{minipage}[b]{0.45\linewidth}
    \centering
        \includegraphics[width=\linewidth]{img/clf128.png}
        \caption{\label{fig:clf128}Arquitetura do classificador do experimento com 128 \textit{MFCCs}}
    \end{minipage}
\end{figure}

% ==========================================================================================
\section{Aplicabilidade}\label{sec:aplicabilidade}

Este trabalho foi desenvolvido com o intuito de explorar possibilidades para inferência da itensidade da emoção na voz em nosso idioma nativo, até então não encontrado na literatura. \textit{SER} é um problema complexo uma vez que a expressão emocional é afetada por vários fatores, incluindo a linguagem falada e o sotaque dos indivúdios ~\cite{6}.

Inferir a intensidade da emoção encontra aplicações potenciais em em diversas áreas, como saúde ~\cite{1}, segurança ~\cite{4}, entretenimento (através de \textit{smart enviromentes} ~\cite{alexa1} e \textit{smart assistants} ~\cite{alexa2}) e comercial\cite{bsignal1}\cite{bsignal2}. Trabalhos como ~\cite{63} buscam entender a intensidade da emoção para melhorar a performance da síntese de uma vocalização emocional oriunda de um mecanismo de \textit{speech to text}.

Desenvolver uma metodologia para classificar a intensidade da emoção poderia colaborar com a monitoração do estado de saúde de pacientes, com um modelo embarcado em assistentes virtuais inteligentes; auxiliar no antendimento ao público encaminhando chamadas de \textit{call centers} compreendidas como urgentes para fora do fluxo automatizado e para o atendimento humano; ser utilizado para melhorar o desempenho de sistemas de segurança baseados em reconhecimento de voz, reduzindo a alteração na voz causada por variações do estado emocional, assim como soluções que reduzem ruído; ou para uma fins comerciais no sistema financeiro, como uma empresa ~\cite{bsignal0} que alega possuir uma solução \textit{From Voice to Revenue} que aumentou em 20\% o sucesso da renegociação para recupração de crédito de liquidação duvidosa.


\section{Experimentos}

Nesta seção apresentaremos, detalhadamente, os resultados dos dois experimentos realizados. Ambos foram submetidos à mesma metodologia e aferição de métrica, o que nos possibilita comparar diretamente o desempenho oriundo das execuções.\\

% \subsection{Primeiro experimento: 64 \textit{MFCC}s}

A Figura \ref{fig:ae64} apresenta a arquitetura do modelo Autoencoder, onde podemos observar que os vetores do espaço latente, portanto, o vetor comprimido para ser posteriormente reconstruído, terão $32$ posições.

Para este modelo, o valor final da \textit{Loss} média ao aplicá-lo ao seu conjunto de teste foi de $6,40$. A Figura \ref{fig:predae64} ilustra uma predição feita pelo Autoencoder: A linha azul representa o dado original (\textit{input});a linha tracejada laranja representa a predição (\textit{output}) daquele registro; e a linha verde foi calculada tomando a diferença entre os coeficientes do \textit{input} e do \textit{output}.

% --- PRED AE 64 --- %
\begin{figure}[h]
    \centering
    \includegraphics[width=0.6\textwidth]{img/_predae64.png}
    \caption{\label{fig:predae64}Exemplo de predição (\textit{encoding e decoding}) do \textit{Autoencoder} para 64 \textit{MFCC}s}
\end{figure}

A Figura \ref{fig:clf64} apresenta a arquitetura do modelo de classificação de intensidade, enquanto a Tabela \ref{table:metricasclf64} apresenta métricas de desempenho para sua aplicação no conjunto de teste. Podemos observar que o melhor resultado para o \textit{F1-Score} deu-se para a intensidade Fraca, com o valor de $0,68$, e ficando abaixo de $0,5$ apenas para a classe Forte.

Aplicamos o classificador aos dados oriundos do $X_{VERBO}$ e anotamos o resultado. Então efetuamos um \textit{PCA} com 2 componentes ao resultado do \textit{encoding} de $X$ (Figura \ref{fig:pca64}), onde as cores representam a classe - original para os dados do $VIVAE$ e predição do classificador para os dados do $VERBO$ - e podemos observar a formação de 2 grandes \textit{clusters}, um a esquerda e outro a direita.

Percebemos através dos \textit{labels}, que estes \textit{clusters} conseguiram distinguir bem as duas bases de dados, bem como um movimento descendente de aumento da intensidade da emoção em ambos os \textit{clusters}. Temos uma visualização mais simples na Figura \ref{fig:pca64-2}, onde agrupamos as classes Fraca e Moderada em uma nova classe denominada Baixa e as classes Forte e Pico em uma classe denominada Alta.

% --- CLF 64 Metrics --- %
\begin{table}[h]\caption{\label{table:metricasclf64}Métricas de avaliação para o classificador do experimento para 64 \textit{MFCCs}}
    \centering
    \begin{tabular}{l|ccc|}
        \cline{2-4}
                                      & \multicolumn{3}{c|}{Métricas}                                                               \\ \hline
            \multicolumn{1}{|l|}{Intensidade} & \multicolumn{1}{c|}{Precision}      & \multicolumn{1}{c|}{Recall}           & F1-Score      \\ \hline
            \multicolumn{1}{|l|}{Fraca}       & \multicolumn{1}{c|}{\textbf{0,73}}  & \multicolumn{1}{c|}{0,64}             & \textbf{0,68} \\ \hline
            \multicolumn{1}{|l|}{Moderada}    & \multicolumn{1}{c|}{0,53}           & \multicolumn{1}{c|}{0,52}             & 0,53          \\ \hline
            \multicolumn{1}{|l|}{Forte}       & \multicolumn{1}{c|}{0,46}           & \multicolumn{1}{c|}{0,50}              & 0,48          \\ \hline
            \multicolumn{1}{|l|}{Pico}        & \multicolumn{1}{c|}{0,64}           & \multicolumn{1}{c|}{\textbf{0,68}}    & 0,66          \\ \hline
        \end{tabular}
\end{table}

\clearpage

% --- PCA 1o exp --- %
% \begin{landscape}
\begin{figure}[h]
    \centering
    \includegraphics[width=1.5\textwidth]{img/pca64.png}
    \caption{\label{fig:pca64}PCA com 2 componentes aplicado ao resultado do \textit{encoding} do primeiro experimento}
\end{figure}
% \end{landscape}

% \begin{landscape}
\begin{figure}[h]
    \centering
    \includegraphics[width=1.5\textwidth]{img/pca64-2.png}
    \caption{\label{fig:pca64-2}Agrupamento em 2 classes para PCA com 2 componentes aplicado ao resultado do \textit{encoding} do primeiro experimento}
\end{figure}
% \end{landscape}

\clearpage

\subsection{Segundo experimento: 128 \textit{MFCC}s}

A Figura \ref{fig:ae128} apresenta a arquitetura do modelo Autoencoder, onde podemos observar que os vetores do espaço latente, portanto, o vetor comprimido para ser posteriormente reconstruído, terão $64$ posições.

Para este modelo, o valor final da \textit{Loss} média ao aplicá-lo ao seu conjunto de teste foi de $0,84$. A Figura \ref{fig:predae128} ilustra uma predição feita pelo Autoencoder: A linha azul representa o dado original (\textit{input}); a linha tracejada laranja representa a predição (\textit{output}) daquele registro; e a linha verde foi calculada tomando a diferença entre os coeficientes do \textit{input} e do \textit{output}.

A Figura \ref{fig:clf128} apresenta a arquitetura do modelo de classificação de intensidade, enquanto a Tabela \ref{table:metricasclf128} apresenta métricas de desempenho para sua aplicação no conjunto de teste. Podemos observar que o melhor resultado para do \textit{F1-Score} deu-se para a intensidade Pico, com o valor de $0,64$, ficando abaixo de $0,5$ apenas para a classe Forte.

Aplicamos o classificador aos dados oriundos do $X_{VERBO}$ e anotamos o resultado. Então efetuamos um \acrshort{PCA} com 2 componentes ao resultado do \textit{encoding} de $X$ (Figura \ref{fig:pca128}), onde as cores representam a classe - original para os dados do $VIVAE$ e predição do classificador para os dados do $VERBO$ - e podemos observar a formação de 2 grandes \textit{clusters}, um a esquerda e outro a direita.

Percebemos através dos \textit{labels}, que estes \textit{clusters} conseguiram distinguir bem as duas bases de dados, bem como um movimento descendente de aumento da intensidade da emoção em ambos os \textit{clusters}. Temos uma visualização mais simples na Figura \ref{fig:pca128-2}, onde agrupamos as classes Fraca e Moderada em uma nova classe denominada Baixa e as classes Forte e Pico em uma classe denominada Alta.

% --- PRED AE 128 --- %
    \begin{figure}%[t]
    \centering
    \includegraphics[width=0.6\textwidth]{img/_predae128.png}
    \caption{\label{fig:predae128}Exemplo de predição (\textit{encoding e decoding}) do \textit{Autoencoder} para 128 \textit{MFCC}s}
\end{figure}

% --- CLF 128 Metrics --- %
\begin{table}%[h]
    \centering
    \begin{tabular}{l|ccc|}
        \cline{2-4}
                                          & \multicolumn{3}{c|}{Métricas}                                                               \\ \hline
        \multicolumn{1}{|l|}{Intensidade} & \multicolumn{1}{c|}{Precision}      & \multicolumn{1}{c|}{Recall}           & F1-Score      \\ \hline
        \multicolumn{1}{|l|}{Fraca}       & \multicolumn{1}{c|}{\textbf{0,62}}  & \multicolumn{1}{c|}{0,55}             & 0,58          \\ \hline
        \multicolumn{1}{|l|}{Moderada}    & \multicolumn{1}{c|}{0,55}           & \multicolumn{1}{c|}{0,48}             & 0,51          \\ \hline
        \multicolumn{1}{|l|}{Forte}       & \multicolumn{1}{c|}{0,49}           & \multicolumn{1}{c|}{0,50}              & 0,49          \\ \hline
        \multicolumn{1}{|l|}{Pico}        & \multicolumn{1}{c|}{0,57}           & \multicolumn{1}{c|}{\textbf{0,73}}    & \textbf{0,64} \\ \hline
    \end{tabular}
        \caption{\label{table:metricasclf128}Métricas de avaliação para o classificador do experimento para 128 \textit{MFCCs}}
\end{table}

% --- PCA 2o exp --- %
\begin{figure}%[h]
    \centering
    \includegraphics[angle=90, width=0.8\textwidth]{img/pca128.png}
    \caption{\label{fig:pca128}PCA com 2 componentes aplicado ao resultado do \textit{encoding} do segundo experimento}
\end{figure}

\begin{figure}%[h]
    \centering
    \includegraphics[angle=90, width=0.8\textwidth]{img/pca128-2.png}
    \caption{\label{fig:pca128-2}Agrupamento em 2 classes para PCA com 2 componentes aplicado ao resultado do \textit{encoding} do segundo experimento}
\end{figure}


\clearpage

\section{Discussão dos resultados}

Nesta seção iremos iniciar uma discussão sobre os resultados alcançados e apresentaremos as dificuldades encontradas, além das limitações gerais do projeto.\\

Com base nos resultados da Tabela \ref{table:resultexp} verificamos que o primeiro experimento obteve desempenho superior no que tange a métrica selecionada para o modelo de classificação de intensidade, tendo um \textit{F1-Score} superior aos do primeiro experimento em três das quatro classes.

Em ambos experimentos, a classe com pior resultado do classificador - de acordo a métrica definida na metodologia - foi semelhante, a classe Forte. E as classes com melhor desempenho são Fraca e Pico de intensidade, respectivamente.

Na Tabela \ref{table:compexp} vemos um ganho de desempenho superior a 7 vezes para o valor da \textit{Loss} no segundo experimento, no qual utilizamos um número maior de \textit{MFCCs}. Embora a \textit{Loss} tenha apresentado uma queda significativa no segundo experimento, o que significa que o \textit{Autoencoder} apresenta um desempenho melhor para reproduzir o dado de entrada, preservando melhor as características e atributos do dado, essa melhora no desempenho de uma função identidade se mostra contraditória frente o desempenho do segundo classificador. Então, podemos supor que uma quantidade estritamente maior de \textit{MFCCs} colaborou para a reconstrução da amostra enquanto não demonstrou ganhos semelhantes na sua utilização para classificar a intensidade da emoção.


% \begin{table}[!h]
%     \centering
%     \begin{tabular}{|l|cc|}
%     \hline
%         \multicolumn{1}{|c|}{\multirow{3}{*}{Intensidade}} & \multicolumn{2}{c|}{MFCCs}                                 \\ \cline{2-3} 
%         \multicolumn{1}{|c|}{}                             & \multicolumn{1}{c|}{64}                & 128               \\ \cline{2-3} 
%         \multicolumn{1}{|c|}{}                             & \multicolumn{1}{c|}{\textit{F1-Score}} & \textit{F1-Score} \\ \hline
%         Fraca                                              & \multicolumn{1}{c|}{\textbf{0,68}}     & 0,58              \\ \hline
%         Moderada                                           & \multicolumn{1}{c|}{\textbf{0,53}}     & 0,51              \\ \hline
%         Forte                                              & \multicolumn{1}{c|}{0,48}              & \textbf{0,49}     \\ \hline
%         Pico                                               & \multicolumn{1}{c|}{\textbf{0,66}}     & 0,64              \\ \hline
%     \end{tabular}
%     \caption{\label{table:resultexp}Comparativo de \textit{F1-Score} entre os experimentos}
% \end{table}

\begin{table}[]
    \centering
    \begin{tabular}{|l|l|l|}
    \hline
        Intensidade & F1-Score para 64 MFCCs & F1-Score para 128 MFCCs \\ \hline
        Fraca & 0,68 & 0,58 \\ \hline
        Moderada & 0,53 & 0,51 \\ \hline
        Forte & 0,48 & 0,49 \\ \hline
        Pico & 0,66 & 0,64 \\ \hline
    \end{tabular}
     \caption{\label{table:resultexp}Comparativo de \textit{F1-Score} entre os experimentos}
\end{table}

% \begin{table}[!h]
%     \centering
%     \begin{tabular}{|l|cc|}
%         \hline
%         \multicolumn{1}{|c|}{\multirow{2}{*}{Atributo do experimento}} & \multicolumn{2}{c|}{MFCCs}         \\ \cline{2-3} 
%         \multicolumn{1}{|c|}{}                                         & \multicolumn{1}{c|}{64}                & 128   \\ \hline
%         Melhor \textit{F1-Score}                                       & \multicolumn{1}{c|}{\textbf{0,68}}     & 0,64  \\ \hline
%         Classe com melhor \textit{F1-Score}                            & \multicolumn{1}{c|}{Fraca}             & Pico  \\ \hline
%         Pior \textit{F1-Score}                                         & \multicolumn{1}{c|}{\textbf{0,48}}     & 0,49  \\ \hline
%         Classe com pior \textit{F1-Score}                              & \multicolumn{1}{c|}{Forte}             & Forte \\ \hline
%         \textit{Loss} de teste (MSE)                                   & \multicolumn{1}{c|}{6,4}               & \textbf{0,84}  \\ \hline
%     \end{tabular}
%     \caption{\label{table:compexp}Comparativo de atributos de desempenho dos experimentos}
% \end{table}

\begin{table}[]
    \centering
    \begin{tabular}{|l|l|l|}
    \hline
        Resultado & MFCCs & ~ \\ \hline
        ~ & 64 & 128 \\ \hline
        Maior F1-Score & 0,68 & 0,64 \\ \hline
        Classe com maior F1-Score & Fraca & Pico \\ \hline
        Pior F1-Score & 0,48 & 0,49 \\ \hline
        Classe com menor F1-Score & Forte & Forte \\ \hline
    \end{tabular}
    \caption{\label{table:compexp}Comparativo de atributos de desempenho dos experimentos}
\end{table}

Embora nossa massa de dados seja composta por poucos registros - quando comparamos com outras bases de dados encontradas na literatura - de posse dos resultados dos experimentos, conseguimos observar que os vetores do espaço latente dos \textit{Autoencoders}, mesmo com uma redução de dimensionalidade à metade, aparentam manter características relativas a intensidade da emoção presentes ambos experimentos. Quando agrupamos as classes Fraca e Moderada no cojunto denominado Baixo e Forte e Pico no conjunto denominado Alto, conseguimos observar nas Figuras \ref{fig:pca64-2} e \ref{fig:pca128-2} que mesmo com o desempenho superior do classificador do primeiro experimento, ainda há uma correspondência entre as intensidades dos registros. Uma vez que os \textit{labels} de $X_{VIVAE}$ são originais, observamos que as classes atribuídas aos dados de $X_{VERBO}$ parecem ser condizentes, vide o comportamento descendente da aumento da intensidade. Assim, o classificador aprendeu com $X_{VIVAE}$ a classificar a intensidade e aplicou essa lógica aos dados em $X_{VERBO}$.
